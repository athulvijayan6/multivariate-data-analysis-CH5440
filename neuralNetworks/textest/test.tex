% @Author: Athul Vijayan
% @Date:   2014-05-08 15:54:08
% @Last Modified by:   Athul Vijayan
% @Last Modified time: 2014-05-08 17:40:03

\documentclass{article}
\usepackage[utf8]{inputenc}
\usepackage[a4paper, margin=1in]{geometry}
\usepackage{amsmath,amssymb,amsthm}
\usepackage{graphicx}

\title{First document}
\author{Hubert Farnsworth \thanks{funded by the ShareLaTeX team}}
\date{February 2014}

\begin{document}

\newcommand{\HRule}{\rule{\linewidth}{0.2mm}} % Defines a new command for the horizontal lines, change thickness here
\begin{titlepage}
\center % Center everything on the page
 
%----------------------------------------------------------------------------------------
%	HEADING SECTIONS
%----------------------------------------------------------------------------------------

\textsc{\LARGE Indian Institute of Technology, Madras}\\[1.5cm] % Name of your university/college
\textsc{\Large Multivariate Data analysis}\\[0.5cm] % Major heading such as course name
\textsc{\large CH5440}\\[0.5cm] % Minor heading such as course title

%----------------------------------------------------------------------------------------
%	TITLE SECTION
%----------------------------------------------------------------------------------------

\HRule \\[0.4cm]
{ \huge \bfseries End Semester Examination}\\[0.4cm] % Title of your document
\HRule \\[1.5cm]
 
%----------------------------------------------------------------------------------------
%	AUTHOR SECTION
%----------------------------------------------------------------------------------------


% If you don't want a supervisor, uncomment the two lines below and remove the section above
\Large \emph{Submitted by:}\\
Athul Vijayan % Your name

ED11B004\\[3cm] % Your name

%----------------------------------------------------------------------------------------
%	DATE SECTION
%----------------------------------------------------------------------------------------

{\large \today}\\[6cm] % Date, change the \today to a set date if you want to be precise
\raggedleft
\includegraphics[width=2cm]{logo.png}

\vfill % Fill the rest of the page with whitespace
\end{titlepage}

\section{Introduction}
Another option is to fill in a rough draft of the text that will appear. This has the advantage of giving you the best preview of the design. Unfortunately, you do not always have the information to do write a good rough draft and unless your rough draft is similar to the final text, the advantage is limited. Furthermore this can be a time-consuming option and if you charge by your time a client may not feel that their money will well spent by having you spend 40 minutes writing a rough draft that will be completely changed.

\begin{equation}
	\begin{align*}
	f(x) &= \sum_{i=1}^{n} \sum_{j=1}^{m} \theta^TX 
	&=\sum_{i=1}^{n} \sum_{j=1}^{m} \theta^TX
	\end{align*}
\end{equation}
\begin{figure}[h]
	\centering
	\includegraphics[width=5cm]{logo.png}
	\caption{IITM LOGO}
\end{figure}

Some designers make up their own text, but there are also horror stories of designers and journalists who accidentally left in their text with embarrassing results. The phrase Dummy Text repeated five times may not impress a client, but it is better than a client seeing text reviewing your neighbor’s hair or suggesting exactly what you want to do with Internet Explorer 6. While you know you would always make sure all of the dummy text has been removed, it is still a good idea to use some common sense with your dummy text.
\section{Theory}
\end{document}